% ========================================================
% This document is a customizable CV/Resume template built using LaTeX.
% The template is designed for easy customization and clear structure.
%
% Author: Matthew DeVerna (www.matthewdeverna.com)
% Date: 2024
% Design: Cased on Hause Lin's CV (hauselin.com)
% 
% Project Overview:
% -----------------
% This LaTeX document is designed to help you create a professional CV or 
% resume with ease. It uses as little fancy LaTex functionality or custom functions as possible to maximize its longterm durability and flexibility.
% The document is structured into multiple sections, each loaded from 
% separate subfiles for modularity and ease of maintenance. 
%
% Key Features:
% -------------
% - Customizable sections: Education, Research Experience, Awards, Publications, etc.
% - Bookmarks in the PDF for easy navigation
% - Styled bibliography with BibLaTeX
% - Hyperlinked email and website
% - FontAwesome icons for additional styling
%
% Getting Started:
% ----------------
% 1. Customize your personal information by modifying the \mytitle command.
% 2. Add your content to the respective subfiles (e.g., education.tex, exp_research.tex).
% 3. Update the bibliography file (ref.bib) with your publications and categorize them with keywords.
%
% Important Notes:
% ----------------
% - This main file includes the overall structure and settings. 
% - Each section has its own detailed instructions for further customization.
%
% ========================================================

\documentclass[11pt]{article} % Choose the document class and font size
\usepackage[margin=1in]{geometry} % Page layout settings

% Set the citation style

\usepackage[
    backend=bibtex,  % Specifies the backend to be used by BibLaTeX for processing the bibliography. 'biber' is the default backend.
    maxnames=20,        % Limits the maximum number of author names to display before abbreviating with "et al."
    style=mla,       % Sets the citation style to 'nature,' which is commonly used in scientific papers.
    sorting=ydnt,       % Specifies the sorting order of entries in the bibliography:
                        % y - year (descending)
                        % d - descending order
                        % n - name
                        % t - title
    defernumbers=true,  % Delays the assignment of citation numbers until the end of the document, allowing for the correct order of citations within each bibliography section.
]{biblatex}
\addbibresource{ref.bib} % Adds the bibliography resource file 'ref.bib' containing all the references.


% Allows columns that stretch across pages
\usepackage{longtable}
%\usepackage{ctex}
% Table functionality and beautification (not strictly needed)
\usepackage{bookmark}



% Use icons, if you want.
% All available icons: http://mirrors.ibiblio.org/CTAN/fonts/fontawesome5/doc/fontawesome5.pdf
\usepackage{fontawesome}

% Allows font justification control (needed for clean pub-list formatting)
\usepackage{ragged2e}

% For underlining with line breaks
\usepackage{soul} 

% All fonts: https://tug.org/FontCatalogue/
\usepackage{kpfonts} % More professional font
% \usepackage[default]{sourcecodepro} % Code-like font
\usepackage[T1]{fontenc}

% Control hyperlinks and colors
% CUSTOM COLORS INCLUDED DIRECTLY AFTER \begin{document}
\usepackage{xcolor}
\usepackage{hyperref}
\hypersetup{
    colorlinks=true,        % Enable colored links
    breaklinks=true,        % Allow links to break across lines
    linkcolor=cornflowerblue,    % Color of internal links
    urlcolor=cornflowerblue,     % Color of URL links
    anchorcolor=cornflowerblue,  % Color of anchors
    citecolor=cornflowerblue,    % Color of citations
    pdftitle={Your Title},    % Title of the PDF
    pdfauthor={Your Name}, % Author of the PDF
    bookmarksopen=true,      % Open bookmarks panel at start
}

%%% CONVENIENCE FUNCTIONS GO HERE %%%
%%% ----------------------------- %%%
\newcommand{\mytitle}[4]{
  \begin{center}
    \Large\textbf{#1}\normalsize \\ % Name in large bold font
    \href{mailto:#2}{#2} \\ % Email with mailto: link
    \href{https://#3}{#3} \\ % Website with link
    #4 % Address
  \end{center}
}
%%% ----------------------------- %%%


\begin{document}
\include{colors.tex} % Load custom colors from colors file
\mytitle{Tianlei Zhang \\ \normalsize{\textnormal{/tee-ahn-lay jahng/}}}{zhangtianlei@whu.edu.cn}{ynbsztl.github.io} % Insert your custom title


% Ensure right side margin is not surpassed by bibliography and the right margin is aligned throughout
\RaggedRight


% These \pdfbookmark lines create bookmarks in the exported PDF document that display in the left pane.
% Value in [] sets the indentation level of the bookmark
\pdfbookmark[1]{Education}{}
\section*{Education}
% Add your educational background here!

% NOTE: If you want to remove the "Expected" footnote, you will want to remove:
% - Directly below: \renewcommand, \setcounter
% - In the table: \footnotemark in the left column
% - After the table: \footnotetext, \renewcommand, \setcounter

% Different numbers in "\setcounter{footnote}{0}" use different symbols
\renewcommand{\thefootnote}{\fnsymbol{footnote}}
\setcounter{footnote}{0}

\begin{longtable}[l]{@{}p{.125\textwidth} p{0.875\textwidth}}
    % Use custom symbol footnote for "expected"
    2022--25\footnotemark & \textbf{M.A.}, Health Economics, Wuhan University \\
    2019-21 & \textbf{B.A.}, Finance, Wuhan University \\

    2017-21 & \textbf{B.E.}, Printing Engineering, Wuhan University \\

    %2008-12 & \textbf{B.Sc.} Field, University
% \end{tabularx}
\end{longtable}


% Add text for the custom footnote
\footnotetext[1]{Expected.}

% Restore the default footnote numbering
\renewcommand{\thefootnote}{\arabic{footnote}}
\setcounter{footnote}{1}


%\pdfbookmark[1]{Research Experience}{exp_research}
%\section*{Research Experience}
%\label{exp_research}
%\input{exp_research.tex}

\subsection*{Research Interests}\label{ri}

\begin{longtable}[l]{@{}p{.1\textwidth} p{0.9\textwidth}}

    \textbf{Area} & Public Policy, Labour Economics, Health Economics \\
    \textbf{Topic} & Disease Control, Fertility Choice  \\
    
\end{longtable}

\pdfbookmark[1]{Publications}{pubs}
\section*{Publications}
\label{pubs}

% Add equal contribution dagger
\vspace{-.75em}
\small
\faGoogle~\href{https://scholar.google.com/citations?hl=zh-CN&user=zsO6zkYAAAAJ}{Google Scholar}\\
%$\dagger \rightarrow$ Equal contribution
\normalsize


\pdfbookmark[2]{Journal Articles}{journal-article}
\subsection*{Journal Articles}
\label{journal-article}
\newrefcontext[labelprefix=J] % Will prefix bibliography numbers with this letter
% Ensures publications which are not cited in the document are included in the above sections
\nocite{*} % Ensures uncited items are included
\printbibliography[
    type=article, % Only include @article ref.bib items
    heading=none, % Do not include header. Gives us more control.
    resetnumbers=true, % Start item counter from zero
    keyword=J % Include items in ref.bib with keyword={J}
]

%\pdfbookmark[2]{Peer-reviewed Conference Proceedings}{conferences}
%\subsection*{Peer-reviewed Conference Proceedings}
%\label{conferences}
%\newrefcontext[labelprefix=C]
%\printbibliography[type=inproceedings,heading=none,resetnumbers=true,keyword=C]

\pdfbookmark[2]{Working papers}{working-papers}
\subsection*{Working Paper}
\label{working-papers}
\newrefcontext[labelprefix=W]
\printbibliography[type=misc,heading=none,resetnumbers=true,keyword=R]

\pdfbookmark[1]{Presentations}{presentations}
\section*{Presentations}
\label{presentations}

% Include any additional details here
% \vspace{-.75em}
% \small
% $\dagger \rightarrow$ Equal contribution
% \normalsize

\pdfbookmark[2]{Talks}{talks}
\subsection*{Talks}
\label{talks}
\newrefcontext[labelprefix=T]
\printbibliography[type=misc,heading=none,resetnumbers=true,keyword=T]


\pdfbookmark[1]{Web Data Crawl}{Web Data Crawl}
\section*{Web Data Crawl}
\label{WebDataCrawl}
% Set custom colors here (imported directly after \begin{document})
% The below use HTML hex codes.
% More HTML hex codes: https://encycolorpedia.com/html
\definecolor{firebrick}{HTML}{b22222} 
\definecolor{darkslategrey}{HTML}{2f4f4f} 
\definecolor{cornflowerblue}{HTML}{6495ed} 
\definecolor{mediumslateblue}{HTML}{7b68ee} 

$^\dagger$ Sample datasets can be downloaded from \href{https://ynbsztl.github.io/data/}{https://ynbsztl.github.io/data/}

\begin{description}
\item[\textbf{Dianping}$^\dagger$] Restaurant review data from a leading application
\subitem 200,000 observations in Wuhan, covering the period from 2022 to 2024
\item[\textbf{People’s Daily}$^\dagger$] Official newspaper of the Chinese Communist Party (CCP)
\subitem Daily reports spanning from 1946 to the present
\item[\textbf{Baidu Index}$^\dagger$] Comparable to Google Trends
\subitem Daily data with nationwide coverage across all cities in China, from 2011 to present
\item[\textbf{Fang Tianxia}$^\dagger$] A real estate website providing comprehensive housing information.
\subitem The dataset includes details such as house prices, community names, addresses, construction years, and other relevant attributes.
\end{description}

% \pdfbookmark[2]{Posters}{posters}
% \subsection*{Posters}
% \label{posters}
% \newrefcontext[labelprefix=P]        
% \printbibliography[type=misc,heading=none,resetnumbers=true,keyword=P]

% \pdfbookmark[2]{Demonstrations \& Tutorials}{demos}
% \subsection*{Demonstrations \& Tutorials}
% \label{demos}
% \newrefcontext[labelprefix=D] \printbibliography[type=misc,heading=none,resetnumbers=true,keyword=D]

\pdfbookmark[1]{Teaching}{teaching}
\section*{Teaching}
\label{teaching}
% \subsection*{Teaching assistant}
% \subsubsection*{Wuhan University}
% % 
% \begin{longtable}[l]{@{}p{.125\textwidth} p{0.875\textwidth}}

%     Fall, 2024 & 2024 MIB quantitative analysis (Master’s). \\
    
%    & Instructor: Prof. Xun Li, Wuhan University  \\
    

% \end{longtable}



\pdfbookmark[1]{Awards \& Honors}{awards}
\section*{Awards \& Honors}
\label{awards}
\input{awards.tex}


\pdfbookmark[1]{Tools \& Software}{tools}
\section*{Tools \& Software}
\label{tools}
\input{tools.tex}





% \pdfbookmark[1]{Selected Media Coverage}{media}
% \section*{Selected Media Coverage}
% \label{media}
% \input{media.tex} % Input lines load the material from the subdocuments





% \pdfbookmark[1]{Academic Advising}{advising}
% \section*{Academic Advising}
% \label{advising}
% \input{advising.tex}


% \pdfbookmark[1]{Academic Service}{service}
% \section*{Academic Service}
% \label{service}
% \input{service.tex}


%\pdfbookmark[1]{Other Experience}{exp_other}
%\section*{Other Experience}
%\label{exp_other}
%\input{exp_other.tex}


% Pretty ending with the date last updated
\centering
\rule{0.25\linewidth}{0.4pt}\\
\medskip
Last updated: \today. \href{https://ynbsztl.github.io/CV_Tianlei.pdf}{[updating version]}

\end{document}
